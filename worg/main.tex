% Created 2022-04-01 Fri 12:28
% Intended LaTeX compiler: pdflatex
\documentclass{report}
                              \usepackage[utf8]{inputenc}
\usepackage[T1]{fontenc}
\usepackage{amsmath,amssymb,array}
\usepackage{booktabs}
\usepackage{natbib}
\usepackage{listings}
\newcommand{\J}{\ensuremath{J}}
\newcommand{\1}{\ensuremath{\mathbf{1}}}
\DeclareMathOperator*{\argmax}{argmax}
\DeclareMathOperator*{\argmin}{argmin}
\newcommand{\h}{\ensuremath{\lambda}}
\newcommand{\indep}{\ensuremath{\perp\hspace*{-1.4ex}\perp}}
\newcommand{\T}{\ensuremath{\widetilde{T}}}
\newcommand{\X}{\ensuremath{{X}}}
\renewcommand{\t}{\ensuremath{\Tilde{t}}}
\newcommand{\ax}{\ensuremath{\mid a,\,{x}}}
\newcommand{\aX}{\ensuremath{\mid A = a,\,{X}}}
\newcommand{\AX}{\ensuremath{\mid A,\,{X}}}
\newcommand{\x}{\ensuremath{{x}}}
\newcommand{\trt}{\ensuremath{\pi^*}}
\newcommand{\tk}{\ensuremath{\tau}}
\newcommand{\lj}{\ensuremath{j}}
\newcommand{\jj}{\ensuremath{k}}
\newcommand{\g}{\ensuremath{\pi}}
\RequirePackage{tcolorbox}
\definecolor{lightGray}{gray}{0.98}
\definecolor{medioGray}{gray}{0.83}
\definecolor{mygray}{rgb}{.95, 0.95, 0.95}
\newcommand{\mybox}[1]{\vspace{.5em}\begin{tcolorbox}[boxrule=0pt,colback=mygray] #1 \end{tcolorbox}}

\lstset{
keywordstyle=\color{blue},
commentstyle=\color{red},stringstyle=\color[rgb]{0,.5,0},
literate={~}{$\sim$}{1},
basicstyle=\ttfamily\small,
columns=fullflexible,
breaklines=true,
breakatwhitespace=false,
numbers=left,
numberstyle=\ttfamily\tiny\color{gray},
stepnumber=1,
numbersep=10pt,
backgroundcolor=\color{white},
tabsize=4,
keepspaces=true,
showspaces=false,
showstringspaces=false,
xleftmargin=.23in,
frame=single,
basewidth={0.5em,0.4em},
}
\renewcommand*\familydefault{\sfdefault}
\itemsep2pt
\author{David Chen, Thomas Gerds, Helene Rytgaard}
\date{}
\title{ConCR-TMLE R Paper}
\begin{document}

\maketitle
\abstract{
An abstract of less than 150 words.
}

\section*{Introduction}
\label{sec:org187e796}

\section*{Data Structure}
\label{sec:org9f0ddac}

Consider a survival analysis on an interval \([0,\,t_{max}]\) with competing risks. Let \(T^a_j\) denote counterfactual time-to-event variables for event \(j\) and intervention \(a\), for competing events \(j \in \mathcal{J} = \{1, 2, \dots, J\}\) and an intervention \(a \in \mathcal{A}\). Our counterfactual data structure can then be denoted by
\[(T^a_j,\;\X\,:\;a\in\mathcal{A},\;j\in\mathcal{J})\]
where \(\X \in \mathbb{R}^d\) is a \(d\)-dimensional vector of baseline covariates. For a single time-point binary intervention, as in many randomized control trials, \(\mathcal{A} = \{0, 1\}\) and the corresponding counterfactual data is 
\[ (T^1_j,\; T^0_j,\;\X\,: \;j\in\mathcal{J})\]

\lstset{language=r,label= ,caption= ,captionpos=b,numbers=none,otherkeywords={}, deletekeywords={}}
\begin{lstlisting}
head(counterfactuals)
\end{lstlisting}

\begin{center}
\begin{tabular}{rrrrrrrr}
 & T.j1.a0 & T.j1.a1 & T.j2.a0 & T.j2.a1 & L1 & L2 & L3\\
\hline
1 & 0.1599887 & 0.4906215 & 0.5399409 & 0.5803671 & -1.7677221 & 4 & 3.0093952\\
2 & 1.1369533 & 1.9210028 & 0.2375033 & 0.9133089 & -0.4916921 & 0 & 0.3294865\\
3 & 0.3447736 & 1.2538906 & 0.4779721 & 0.8540658 & 0.3214659 & 3 & 4.1630246\\
4 & 4.6631762 & 0.3718961 & 1.5650534 & 0.2485393 & 1.4606608 & 3 & 1.5313713\\
5 & 0.1430018 & 0.5951058 & 0.3003895 & 0.9765322 & 1.5372426 & 2 & 1.5580743\\
6 & 1.8419819 & 3.9131870 & 1.8517334 & 3.0117075 & -0.3395685 & 4 & 0.8455748\\
\end{tabular}
\end{center}

Let \(O\) denote the corresponding coarsened observed data where \(O \sim P_0\). The observed data would include the time-to-censoring \(C\), and observed intervention \(A\). The time to first event (censoring or otherwise) we denote as \(\T = \min(C,\; T_j\,: \; j \in \mathcal{J})\) with \(\Delta = (\argmin\limits_j T_j) \times \1(\min\limits_j T_j \leq C)\) marking which outcome is observed (\(\Delta = 0\) being that censoring occurred). The observable right-censored survival data with competing events can then be represented as 
\[O = (\T,\;\Delta,\;A,\;\X)\]

\lstset{language=r,label= ,caption= ,captionpos=b,numbers=none,otherkeywords={}, deletekeywords={}}
\begin{lstlisting}
head(observed)
\end{lstlisting}

\begin{center}
\begin{tabular}{rrrrrrr}
 & T.tilde & Delta & A & L1 & L2 & L3\\
\hline
1 & 0.20711055 & 0 & 1 & -1.7677221 & 4 & 3.0093952\\
2 & 0.91147298 & 0 & 1 & -0.4916921 & 0 & 0.3294865\\
3 & 0.08374201 & 0 & 0 & 0.3214659 & 3 & 4.1630246\\
4 & 0.29772679 & 0 & 0 & 1.4606608 & 3 & 1.5313713\\
5 & 0.14300179 & 1 & 0 & 1.5372426 & 2 & 1.5580743\\
6 & 1.06839386 & 0 & 0 & -0.3395685 & 4 & 0.8455748\\
\end{tabular}
\end{center}

This observed data also allows the ``long-format'' formulation, where with single time-point intervention variable \(A\) and baseline covariate vector \(\X\), \[O = (N_j(t),\;N_c(t),\;A,\;\X\,:\, j\in\mathcal{J}, t \leq \T)\] Here \(N_j(t) = \1(\T \leq t, \Delta = j)\) and \(N_c(t) = \1(\T \leq t, \Delta = 0)\) denote counting processes for event \(j\) and censoring respectively.

Under coarsening at random (CAR), the observed data likelihood can be factorized as
\begin{align*}
p(O) = p&(\X)\, \g(A \mid \X)\, \lambda_c(\T \AX)^{\1(\Delta = 0)} S_c(\T\text{-} \AX)\\
&\prod_{j=1}^{J} S(\T\text{-} \AX) \, \lambda_j(\T \AX)^{\1(\Delta = j)}
\end{align*}
where \(\lambda_c(t \AX)\) is the hazard of the censoring process and \(\lambda_j(t \AX)\) is the hazard of the \(j^{th}\) event process. Additionally
\begin{align*}
    S_c(t \ax) &= \exp\left(-\int_{0}^{t} \lambda_c(s \ax) \,ds\right)
\intertext{while in a pure competing risks setting}
    S(t \ax) &= \exp\left(-\int_{0}^{t} \sum_{j=1}^{J} \lambda_j(s \ax) \,ds\right)
\intertext{and} 
    F_j(t \ax) &= \int_{0}^{t} S(s\text{-} \ax) \lambda_j(s \ax)\,ds\\
    &= \int_{0}^{t} \exp\bigg(-\int_{0}^{s} \sum_{j=1}^{J} \lambda_j(u \ax)\,du\bigg) \lambda_j(s \ax)\,ds.
\end{align*}

\section*{Target Parameter}
\label{sec:org1e6b693}
Given the identification assumptions of
\begin{enumerate}
\item Consistency : \(T = T^a\) when \(A = a\) for $a = 0,1$.
\item No unmeasured confounding: \(T^a \indep A \mid \X\) for $a = 0,1$.
\item Coarsening at random on censoring: \(T \indep C \AX\)
\end{enumerate}
the hypothetical distribution for data generated following a desired treatment regime involving \(A \sim \trt(A \mid \X)\) and the prevention of the censoring process can be identified as
\[p^{\trt}(O) = p(\X)\, \trt(A \mid \X)\, \prod_{j=1}^{J} S(\T\text{-} \AX) \lambda_j(\T \AX)^{\1(\Delta = j)}\]
For a target parameter of the cause \(\jj \in \J\) absolute risk at time \(\tk\) under this treatment regime \(\trt\), the corresponding efficient influence function \(D^{*}_{\trt, \jj, \tk}(P)(O)\) is
\begin{align*}
    \sum_{j = 1}^{J} \int_{0}^{\tk} \bigg[&\frac{\trt(A \mid \X)\, \1(s \leq \tk)}{\g(A \mid \X)\,S_c(s\text{-} \AX)} \left(\1(\delta = \jj) - \frac{F_\jj(\tk \AX) - F_\jj(s \AX)}{S(s \AX)}\right) \\
    & \hspace{5.2cm} \left(N_j(ds) - \1(\T \geq s)\,\lambda_j(s \AX)\right) \bigg] \,ds\\[2mm]
    &\hspace{1cm}+ \sum_{a=0,1} F_\jj(t \mid A = a, \X)\,\trt(a \mid X) - \Psi_{\trt, \jj, \tk}(P_0)
\end{align*}
with a clever covariate  \(h_{\trt, \jj, j, \tk, s}\)
\begin{align*}
h_{\trt,\, \jj,\, j,\, \tk}(s) = \frac{\trt(A \mid \X)\, \1(s \leq \tk)}{\g(A \mid \X) S_c(s\text{-} \AX)} \left(\1(\delta = \jj) - \frac{F_\jj(\tk \AX) - F_\jj(s \AX)}{S(s \AX)}\right)
\end{align*}
In the binary point treatment case, for the cause \(\jj\) absolute risk at time \(\tk\) if all individuals had been assigned to the treatment condition, \(\trt = (A = 1)\), we would have 
\begin{align*}
    D^{*}_{1, \jj, \tk}(P)(O) =\\
\sum_{j = 1}^{J} \int_{0}^{\tk} \bigg[&\frac{\1(A =1)\, \1(s \leq \tk)}{\g(A \mid \X)\,S_c(s\text{-} \AX)} \left(\1(\delta = \jj) - \frac{F_\jj(\tk \AX) - F_\jj(s \AX)}{S(s \AX)}\right) \\
    & \hspace{3cm} \left(N_j(ds) - \1(\T \geq s)\,\lambda_j(s \AX)\right) \bigg] \,ds\\[2mm]
    &\hspace{1cm}+ F_\jj(t \mid A = 1,\, \X) - \Psi_{\trt, \jj, \tk}(P_0)
\intertext{with a clever covariate  $h_{\trt, \jj, j, \tk, s}$}
h_{1,\, \jj,\, j,\, \tk}(s) = &\frac{\1(A = 1)\, \1(s \leq \tk)}{\g(A \mid \X) S_c(s\text{-} \AX)} \left(\1(\delta = \jj) - \frac{F_\jj(\tk \AX) - F_\jj(s \AX)}{S(s \AX)}\right)
\end{align*}

For estimation of survival-curve derived estimands such as the cause-specific absolute risks, the components of the data distribution that must be estimated are \(g(A \mid \X)\) and \(S_c(t \AX)\), \(\lambda_j(t \AX)\), \(F_j(t \AX)\), and \(S(t \AX)\)

\section*{Estimation}
\label{sec:org297e51f}
\subsection*{Cross-Validation Specification}
\label{sec:orgdbd96c1}
Let \(D_n = \{O_i\}_{i=1}^n\) be an observed sample of \(n\) i.i.d observations of \(O \sim P_0\). For \(V\text{-fold}\) cross validation, let \(B_n = \{1, ... , V\}^n\) be a random vector that assigns the \(n\) observations into \(V\) validation folds. For each \(v \in \{1, ..., V\}\) we then define training set \(D^\mathcal{T}_v = \{O_i : B_n(i) = v\}\) with the corresponding validation set \(D^\mathcal{V}_v = \{O_i : B_n(i) \neq v\}\).

\subsubsection*{Stratified Cross-Validation}
\label{sec:org9b4bf4d}
\lstset{language=r,label= ,caption= ,captionpos=b,numbers=none,otherkeywords={}, deletekeywords={}}
\begin{lstlisting}
StrataIDs <- factor(paste(observed[["A"]], observed[["Delta"]]))
CVFolds <- origami::make_folds(n = observed,
			       fold_fun = origami::folds_vfold,
			       strata_ids = StrataIDs)
\end{lstlisting}

\subsection*{Propensity Score Estimation}
\label{sec:orgbc93f3e}
For the true conditional distribution of \(A\) given \(\X\), \(\pi_0(\cdot \mid \X)\), and \(\Hat{\pi} : D_n \to \Hat{\pi}(D_n)\), let \(L_\pi\) be a loss function such that the risk \(\mathbb{E}_0\left[L_\pi(\Hat{\pi}, O)\right]\) is minimized when \(\Hat{\pi} = \pi_0\). For instance, with a binary \(A\), we may specify the negative log loss \(L_\pi(\Hat{\pi}, O) = \text{-}\log\left(\Hat{\pi}(1 \mid \X)^A \; \Hat{\pi}(0 \mid \X))^{1-A}\right)\). We can then define the discrete superlearner selector which chooses from a set of candidate models \(\mathcal{M_\pi}\) the candidate propensity score model that has minimal cross validated risk 
\[ \Hat{\pi}^{SL} = \argmin_{\Hat{\pi} \in \mathcal{M}_\pi} \sum_{v = 1}^{V} P_{D^\mathcal{V}_v} \; L_\pi(\Hat{\pi}(D^\mathcal{T}_v), D^\mathcal{V}_v)\]

This discrete superlearner model \(\Hat{\pi}^{SL}\) is then fitted on the full observed data \(D_n\) and used to estimate \(\pi_0(A \mid \X)\)

\lstset{language=r,label= ,caption= ,captionpos=b,numbers=none,otherkeywords={}, deletekeywords={}}
\begin{lstlisting}
CovDataTable <- observed[, -c("T.tilde", "Delta", "A")]
Models <- list("Trt" = sl3::make_learner(sl3:::Lrnr_glm))
Intervention <- list(
  "A=1" = list("intervention" = function(a, L) rep_len(1, length(a)),
	       "g.star" = function(a, L) {as.numeric(a == 1)}),
  "A=0" = list("intervention" = function(a, L) rep_len(0, length(a)),
	       "g.star" = function(a, L) {as.numeric(a == 0)})
)

RegsOfInterest <- getRegsOfInterest(Intervention = Intervention,
				    Treatment = observed[["A"]],
				    CovDataTable = CovDataTable)

PropScores <- getPropScore(Treatment = observed[["A"]],
			   CovDataTable = CovDataTable,
			   Models = Models,
			   MinNuisance = 0.05,
			   RegsOfInterest = RegsOfInterest,
			   PropScoreBackend = "sl3",
			   CVFolds = CVFolds)
\end{lstlisting}

\subsection*{Hazard Estimation}
\label{sec:org85ca41d}
Let \(\lambda_{0,\,\delta}\) be the true censoring and cause-specific hazards when \(\delta = 0\) and \(\delta = 1, \dots, J\) respectively. Let \(\mathcal{M}_\delta\) for \(\delta = 0, \dots, J\) be the sets of candidate models, \(\{\Hat{\lambda}_\delta : D_n \to \Hat{\lambda}_\delta(D_n)\}\), for the censoring and cause-specific hazards and let \(L_\delta\) be loss functions such that the risks \(\mathbb{E}_0\left[L_\delta(\Hat{\lambda}_\delta, O)\right]\) are minimized when \(\Hat{\lambda}_\delta = \lambda_{0,\,\delta}\), for instance log likelihood loss. We can then define the discrete superlearner selectors for each \(\delta\) which choose from the set of candidate models \(\mathcal{M_\delta}\) the candidate propensity score model that has minimal cross validated risk 
\[ \Hat{\lambda}_\delta^{SL} = \argmin_{\Hat{\lambda}_\delta \in \mathcal{M}_\delta} \sum_{v = 1}^{V} P_{D^\mathcal{V}_v} \; L_\pi(\Hat{\lambda}_\delta(D^\mathcal{T}_v), D^\mathcal{V}_v)\]

These discrete superlearner selections \(\Hat{\lambda}_\delta^{SL}\) are then fitted on the full observed data \(D_n\) and used to estimate \(\lambda_\delta(t \AX), \, F_\delta(t \AX),\, S(t \AX), \text{ and } S_c(t\text{-} \AX)\) for \(j = 1,\dots, J\).


\subsubsection*{Lagged Censoring Survival}
\label{sec:org4343500}
Let \(\mathcal{S} = \{s_1, s_2, \dots, s_m\}\) be the set containing all target and observed event times, ordered such that \(s_1 < s_2 < \dots < s_m\) . Then for all \(s \,\in\, \mathcal{S}\) we compute
\[ \Hat{S}_c(s\text{-} \AX) = \prod_{s_i < s} \left(1 - \Hat{\lambda}_0^{SL}(s_{i} \AX)\right) \]

\subsubsection*{Cause-Specific Hazards, Event-Free Survival, and Cause-Specific Absolute Risks}
\label{sec:orgefd1247}
For \(\lj = 1,\dots,J\) and \(s \,\in\, \mathcal{S}\), the super learner selections \(\Hat\lambda_\lj^{SL}\) are fit on the full observed data \(D_n\), and used to compute the event free survival
\begin{align*}
\Hat S(s \AX) &= \exp\left(\text{-} \sum_{s_i \leq s} \sum_{j = 1}^{J} \Hat\lambda^{SL}_j(s_i \AX) \right)
\intertext{cause-specific absolute risks}
\Hat F_\lj(s \AX) &= \sum_{s_i \leq s} \Hat S(s_i \AX) \, \Hat\lambda^{SL}_\lj(s_i \AX)
\end{align*}

\section*{Computing the Efficient Influence Function}
\label{sec:org2007f1e}
For each desired treatment regime \(\trt\), each target time \tk, and each target event \jj, the efficient influence functions for each individual are computed in parts.

\subsection*{Clever Covariate: Nuisance Weight}
\label{sec:orgde8c927}
For every \(s_i \,\in\, \mathcal{S}\)
\begin{align*}
     NW_i = \frac{1}{\g(a \mid \x) S_c(s_i\text{-} \ax)}
\end{align*}

1 nuisance weight for every individual at every time \(s_i \,\in\, \mathcal{S}\)

\subsection*{Clever Covariate \(h_{\trt, \jj, j, \tk}(s_i)\)}
\label{sec:org8414fb0}
The stored cause-specific hazards \(\Hat\lambda^{SL}_\lj(s_i \ax)\) and event-free survival \(\Hat S(s_i \ax)\) are used to calculate the cause-specific absolute risks \(\Hat F_\lj(s_i \ax)\), then combined with the nuisance weight to calculate the clever covariates.
\begin{align*}
    h_{\trt,\, \jj,\, j,\, \tk}(s_i) = \trt(a \mid \x)\, \1(s_i \leq \tk)
\times \text{NW}_i \times \left(\1(\delta = \jj) - \frac{F_\jj(\tk \ax) - F_\jj(s_i \ax)}{S(s_i \ax)}\right)
\end{align*}

1 clever covariate for every individual, for every regime of interest, for every target event, for every target time, at every time \(s_i \,\in\, \mathcal{S}\).

\subsection*{EIC}
\label{sec:org65b4a43}
The sum over events and over time are done in a per person loop, the addition of the absolute risk and subtraction of the target estimand are done later, outside of the loop.
\begin{align*}
    \sum_{j = 1}^{J} \int_{0}^{\tk \wedge \Tilde{t}} \;  h_{\trt,\, \jj,\, j,\, \tk}(s) &\times \left(N_j(ds) - \1(\T \geq s)\,\lambda_j(s \AX)\right) \,ds\\[2mm]
    &{\color{red}+ F_\jj(t \mid A = \trt,\,\X) - \Psi_{\trt, \jj, \tk}(P_0)}
\end{align*}
\end{document}